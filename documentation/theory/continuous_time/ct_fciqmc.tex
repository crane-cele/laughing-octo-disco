\documentclass[a4paper, 11pt]{article}

\usepackage{amsmath}
\usepackage{mathtools}
\usepackage{enumerate}

\newcommand{\bi}{\mathbf{i}}
\newcommand{\bj}{\mathbf{j}}
\newcommand{\bk}{\mathbf{k}}
\newcommand{\bz}{\mathbf{0}}
\newcommand{\br}{\mathbf{r}}
\newcommand{\bra}{\langle}
\newcommand{\ket}{\rangle}
\newcommand{\dd}[2]{\frac{d#1}{d#2}}
\newcommand{\Hamil}{\hat{H}}

\newcommand{\sgn}{\operatorname{sgn}}

\begin{document}

\title{Continuous-Time FCIQMC}
\author{J.S. Spencer\\
  Departments of Physics and Materials\\
  Imperial College London 
  \and W.M.C. Foulkes\\
  Department of Physics
  Imperial College London}

\maketitle

\section*{FCIQMC as a Rate Process}

The psip population in FCIQMC is evolved in imaginary time using a series of time steps, where each time step is of length $\tau$.  $\tau$ is set to be small to ensure convergence onto the lowest eigenstate of the Hamiltonian and so smaller values have to be used as the system size increases due to (typically) a larger spread in eigenvalues.  A consequence of this is that the vast majority of attempted spawning steps are rejected, thus wasting a lot of CPU time.  Here we describe how to perform FCIQMC without a time step by `jumping' to the next spawning event.

We start from the standard diffusion equation used in FCIQMC, namely:
\begin{equation}
\dd{c_\bi}{t} = - \sum_\bj T_{\bi\bj} c_\bj
\end{equation}
where
\begin{equation}
T_{\bi\bj} = H_{\bi\bj} - (\bra D_\bz | \Hamil | D_\bz \ket + S)\delta_{\bi\bj}.
\end{equation}
We define the matrices $\mathbf{R}$ and $\boldsymbol{\Xi}$ by
\begin{gather}
R_{\bi\bj} = |T_{\bi\bj}| \\
\Xi_{\bi\bj} = \sgn(T_{\bi\bj})
\end{gather}
and so the diffusion equation can be written as
\begin{equation}
\dd{c_\bi}{t} = - \sum_\bj \Xi_{\bi\bj} R_{\bi\bj} c_\bj.
\end{equation}
Consider a psip that is on determinant $\bj$.  The probability that this psip spawns a child onto determinant $\bi$ in time $\Delta t$ is $R_{\bi\bj}\Delta t$.  Hence $R_{\bi\bj}$ is the \emph{rate} at which a psip on determinant $\bj$ spawns children on determinant $\bi$.  A psip on determinant $\bj$ may spawn children on all other determinants $\{\bi\}$ (where $\{\bi\}$ includes $\bj$, i.e.\ spawning onto the same determinant is permitted), with each spawning connection occurring at its own rate.

What is the probability, $p_\bi$, that the first child spawned appears
on determinant $\bi$ during the time step starting at time $t=n\Delta t$?
\begin{align}
\begin{split}
p_\bi =& 
\underbrace{\prod_{\bk} (1-R_{\bk\bj}\Delta
  t)}_{\textrm{\parbox{4cm}{prob.\ \emph{no} children spawned in 1st
      time step}}} \times
\quad\underbrace{\prod_{\bk} (1-R_{\bk\bj}\Delta t)}_{\textrm{\parbox{4cm}{prob.\ \emph{no} children spawned in 2nd time step}}}
\cdots \\
& \times \underbrace{\prod_{\bk} (1-R_{\bk\bj}\Delta t)}_{\textrm{\parbox{4cm}{prob.\ \emph{no} children spawned in $n$-th time step}}} 
\quad\times \underbrace{R_{\bi\bj}\Delta
  t}_{\textrm{\parbox{4cm}{\rule{0mm}{5.5mm} prob.\ child spawned on determinant $\bi$ in $(n+1)$-th time step}}}
\end{split} \\
\hphantom{p}=& \prod_{\bk} (1-R_{\bk\bj}\Delta t)^n R_{\bi\bj}\Delta t \\
\hphantom{p}=& \prod_{\bk} 
\left(1-\frac{R_{\bk\bj}t}{n}\right)^n R_{\bi\bj}\Delta t .
\end{align}
Using the limit definition of the exponential function, 
\begin{equation}
e^x = \lim_{n\to\infty} \left(1+\frac{x}{n}\right)^n,
\end{equation}
this becomes:
\begin{align}
p_\bi &= \prod_{\bk} \exp\left(-R_{\bk\bj}t\right) R_{\bi\bj}\Delta t \\
  &= \exp\left(-\sum_\bk R_{\bk\bj}t\right)  R_{\bi\bj}\Delta t.
\end{align}
Thus
\begin{eqnarray}
& & p(\textrm{1st spawning is on determinant $\bi$ between times $t$ and
 $t+dt$}) \hspace*{3em} \\\nonumber
&& \quad = \;\; e^{-R_{\bj}t} R_{\bi\bj} dt,
\end{eqnarray}
where we define $R_\bj=\sum_\bk R_{\bk\bj}$. Summing over final states
$\bi$ gives:
\begin{equation}
p(\textrm{1st spawning is between times $t$ and $t+dt$}) = e^{-R_\bj t} R_\bj dt.
\end{equation}
We note that the probability is normalised:
\begin{equation}
\int_0^\infty R_\bj e^{-R_\bj t} dt = 1.
\end{equation}

We can thus evolve the population dynamics by `jumping' to the next spawning event.  For each psip (on, say, determinant $\bj$):
\begin{enumerate}
\item Pick next spawning time from the probability distribution function $f_t(t) = R_\bj e^{-R_\bj t}$.  A point can be sampled from $f_t(t)$ as follows:
    \begin{enumerate}[a)]
    \item Choose a random number, $u$, from the uniform distribution on $[0,1)$. $f_u(u) du = du$ gives the probability that $u$ lies between $u$ and $u+du$.
    \item Calculate $t$ from
    \begin{equation}
    t = - \frac{1}{R_\bj} \ln(u)
    \end{equation}
    so that $0 \le t < \infty$. 
    \end{enumerate}
Then as $f_t(t) |dt| = f_u(u) |du|$, it follows that $f_t(t) = R_\bj e^{-R_\bj t}$ as required.
\item Once the spawning time, $t$, has been chosen, choose the determinant $\bi$ on which to spawn according to
\begin{equation}
p_\bi = \frac{R_{\bi\bj}}{R_\bj}.
\end{equation}
This can be done simply by evaluating all the non-zero $p_\bi$ values and using them to tile the interval $[0,1]$.  If a random number chosen from the uniform distribution on $[0,1]$ lies on segment $\bi$, then the spawning event occurs on determinant $\bi$.
\end{enumerate}

This process makes it possible to perform FCIQMC without time steps and no rejections, hence leading to efficient sampling.  In the Hubbard model the summations $R_\bi$ are easy and fast to evaluate.

However, each spawning event happens at different times for each psip.  How can annihilation (which is crucial to avoid an exponential growth in noise) occur?  We can do this by introducing `annihilation barriers'.

\begin{itemize}
\item Choose an annihilation time step, $t_a$.  Since annihilation is costly, it might be sensible to make this reasonably large.  However, if it is too large then the memory demands are that much greater and (more importantly) the noise in the simulation increases dramatically.  A time step such that around $40-50\%$ of psips spawn between annihilation barriers might be a good value.
\item Advance all psips, spawning event by spawning event, until the next event for that psip happens \emph{after} $t_a$.  As soon as this happens, advance the psip to time $t_a$ but do not carry out the spawning event which happens after $t_a$.  Note: we also have to advance all children to the annihilation barrier (and the grandchildren and great-grandchildren etc.), which leads to an increase in memory as $t_a$ increases.  If it is easy to annihilate pairs of psips during the time evolution (e.g.\ if a psip spawns a child of the opposite sign on its own determinant) then do so.
\item Once all psips in the simulation have reached $t_a$, annihilation is performed in exactly the same fashion as in standard FCIQMC.
\item After annihilation, start advancing all remaining psips to the next annihilation barrier at time $2t_a$.
\end{itemize}

Why does introducing annihilation barriers work?  First, consider a psip on determinant $\bj$ at time $t<t_a$.  If we ignore the annihilation barrier for a moment, then the probability of the next spawning event taking place between times $t^\prime$ and $t^\prime+dt^\prime$, where $t^\prime$ is after the annihilation barrier (i.e.\ $t^\prime>t_a$) is simply
\begin{equation}
R_\bj e^{-R_\bj (t^\prime - t)} dt^\prime.
\end{equation}
With the barrier present, the probability that the next spawning event occurs during the interval $[t^\prime,t^\prime+dt^\prime]$ is given by:
\begin{align}
\begin{split}
\left(\textrm{\parbox{3cm}{probability we pick a time $t_1>t_a$, so advancing to the barrier}}\right) &\times \left(\textrm{\parbox{3cm}{probability the first event after the barrier happens during the interval$[t^\prime,t^\prime+dt^\prime]$}}\right) \\
&=  \left(\int_{t_1=t_a}^\infty R_\bj e^{-R_\bj(t_1-t)} dt_1\right) \times \left(R_\bj e^{-R_\bj(t^\prime-t_a)}dt^\prime\right)
\end{split} \\
&= e^{-R_\bj(t_a-t)} R_\bj e^{-R_\bj(t^\prime-t_a)} dt^\prime \\
&= e^{-R_\bj(t^\prime-t)} dt^\prime.
\end{align}
Thus the introduction of the annihilation barrier has no effect on the statistics of the sequence of spawning events!

\section*{Continuous-Time FCIQMC with Model Rates}

For simple Hamiltonians such as the Hubbard model it may appear
practical to evaluate all of the non-zero spawning rates $R_{\bi\bj}$ of
a psip on configuration $\bj$ and hence the total rate $R_{\bj} =
\sum_{\bi} R_{\bi\bj}$ at which that psip spawns progeny. In reality,
however, even for the Hubbard model, although the non-zero off-diagonal
rates are all identical in magnitude, evaluating the total number of
non-zero off-diagonal matrix elements quickly is problematic.  In
interesting cases, the ct-fciqmc algorithm described above is far too
slow: the connectivity of the space of configurations is so high and the
number of off-diagonal matrix elements so large that we cannot waste
time checking them all.

\subsection*{Algorithm described in terms of model rates}

We can circumvent this problem by using simple model rates
$\tilde{R}_{\bi\bj}$ in place of the exact rates. For the time being (it
may be possible to get around this limitation) we assume that the model
rate $\tilde{R}_{\bi\bj}$ at which a psip on $\bj$ spawns progeny on
$\bi$ is always greater than or equal to the exact rate:
\begin{equation*}
\tilde{R}_{\bi\bj} \geq R_{\bi\bj} .
\end{equation*}
This, of course, requires the model rate to be non-zero whenever the
exact rate is non-zero. If a ct-fciqmc psip is at configuration $\bj$ at
time $t$, the spawning algorithm is as follows:
\begin{enumerate}
\item Work out the model rates $\tilde{R}_{\bi\bj}$ for spawning at
  every configuration $\bi$ connected (in the model) to configuration
  $\bj$. Since $\tilde{R}_{\bi\bj} \geq R_{\bi\bj} \; \forall \bi, \bj$,
  every configuration $\bi$ connected to $\bj$ in the real system is
  also connected in the model.
\item Evaluate the total model spawning rate from configuration $\bj$:
\begin{equation*}
  \tilde{R}_{\bj} = \sum_{\bi} \tilde{R}_{\bi\bj} .
\end{equation*}
\item Pick a spawning time $t'$ from the model Poisson distribution:
\begin{equation*}
  \tilde{R}_{\bj} e^{-\tilde{R}_{\bj}(t'-t)} .
\end{equation*}
\item Pick a spawning site $\bi$ from the probability distribution
\begin{equation*}
  \tilde{p}_{\bi} = \frac{\tilde{R}_{\bi\bj}}{\tilde{R}_{\bj}} .
\end{equation*}
\item \emph{Accept} the spawning event with probability
\begin{equation*}
  \frac{R_{\bi\bj}}{\tilde{R}_{\bi\bj}} .
\end{equation*}
(Since $\tilde{R}_{\bi\bj} \geq R_{\bi\bj}\;\forall \bi,\bj$, the
acceptance probability is always $\leq 1$.)  If the spawning event is
\emph{rejected}, advance the simulation clock to $t'$ but do not spawn.
\end{enumerate}
The algorithm is identical to the standard ct-fciqmc algorithm as far as
step 4, except that the exact rates are replaced by the model rates. The
rejection step described in 5 corrects for this replacement. The
calculation of $\tilde{R}_{\bj}$ requires all of the \emph{model}
spawning rates from $\bj$, but the only \emph{exact} rate required when
a psip at $\bj$ attempts to spawn a child at $\bi$ is $R_{\bi\bj}$.

Taking the rejection step into account, the probability that the first
spawning event attempted by the psip on configuration $\bj$ at time
$t$ creates a new psip on configuration $\bi$ in the time interval
$[t',t'+dt']$ is
\begin{align*}
  \MoveEqLeft (\text{prob.\ spawning on $\bi$ attempted}) \times
  (\text{prob.\ spawning on $\bi$ accepted}) \\
  & = \left ( \tilde{R}_{\bj} e^{-\tilde{R}_{\bj}(t' - t)} dt' \; \times \;
  \frac{\tilde{R}_{\bi\bj}}{\tilde{R}_{\bj}} \right ) \; \times \; 
  \left ( \frac{R_{\bi\bj}}{\tilde{R}_{\bi\bj}} \right ) \\
  & = R_{\bi\bj} e^{-\tilde{R}_{\bj}(t-t')} dt' .
\end{align*}
At first sight this result appears wrong, since the model total spawning
rate $\tilde{R}_{\bj}$ appears in the exponent instead of the exact
total spawning rate $R_{\bj}$. Progeny appear on sites $\bi$ with the
correct \emph{relative} rates $R_{\bi\bj}$ but the wrong total rate. We
will come back to this point later, explaining why this is in fact
exactly what we want to happen.

The \emph{total} probability that a spawning event is attempted in the
interval $[t',t'+dt']$ and rejected is
\begin{align*}
  \MoveEqLeft \sum_{\bi} (\text{prob.\ spawning on $\bi$ attempted}) \times
  (\text{prob.\ spawning on $\bi$ rejected}) \\
  & = \sum_{\bi} \left (  \tilde{R}_{\bi\bj} e^{-\tilde{R}_{\bj}(t'-t)} dt' 
  \right ) \times \left ( 1 -
    \frac{R_{\bi\bj}}{\tilde{R}_{\bi\bj}} \right ) \\
  & = \left ( e^{-\tilde{R}_{\bj}(t'-t)} dt'\right ) \times
  \sum_{\bi} \left ( \tilde{R}_{\bi\bj} - R_{\bi\bj} \right ) \\
  & = \left ( \tilde{R}_{\bj} - R_{\bj} \right ) e^{-\tilde{R}_{\bj}(t'-t)}
  dt' .
\end{align*}

\subsection*{Algorithm described in terms of null events}

To understand why this algorithm works, it helps to change point of view
slightly. Instead of thinking in terms of model rates
$\tilde{R}_{\bi\bj}$, consider instead a ct-fciqmc algorithm that uses
the \emph{exact} rates $R_{\bi\bj}$ but includes the possibility of
additional \emph{null} spawning events, which occur at rate
$\tilde{R}_{\bj} - R_{\bj}$. If a null spawning event takes place, the
simulation clock is advanced to the spawning time but no new psips are
created, just as if a spawning attempt has been rejected in the
model-rates algorithm.

The total rate of spawning, including null events, is
\begin{equation*}
  \left ( \sum_{\bi} R_{\bi\bj} \right ) 
  + \left ( \tilde{R}_{\bj} - R_{\bj} \right ) 
  = R_{\bj} + \left ( \tilde{R}_{\bj} - R_{\bj} \right ) = \tilde{R}_{\bj}.
\end{equation*}
The probability that the first spawning event takes place in the time
interval $[t',t'+dt']$ and produces a psip on configuration $\bi$ is
\begin{equation*}
  \tilde{R}_{\bj} e^{-\tilde{R}_{\bj}(t'-t)} dt' \times
  \frac{R_{\bi\bj}}{\tilde{R}_{\bj}} = R_{\bi\bj}
  e^{-\tilde{R}_{\bj}(t'-t)} dt',
\end{equation*}
and the probability that the first spawning event takes place in the
time interval $[t',t'+dt']$ and does nothing except advance the clock
(i.e., is a null event) is
\begin{equation*}
  \tilde{R}_{\bj} e^{-\tilde{R}_{\bj}(t'-t)} dt' \times 
  \frac{\tilde{R}_{\bj} - R_{\bj}}{\tilde{R}_{\bj}} =
  \left ( \tilde{R}_{\bj} - R_{\bj} \right ) e^{-\tilde{R}_{\bj}(t'-t)}
  dt' .
\end{equation*}
These rates are exactly as in the model-rates approach, so the two
algorithms are equivalent.  I find the formulation in terms of exact
rates and null spawning the more natural of the two and will use it from
now on.

\subsection*{Why does the model-rates/null-events algorithm work?}

Suppose that there is a psip on configuration $\bj$ at time $t$.  As
explained above, the probability that the first spawning event attempted
by this psip creates a new psip on configuration $\bi$ in the time
interval $[t',t'+dt']$ is
\begin{equation*}
R_{\bi\bj} e^{-\tilde{R}_{\bj}(t'-t)} dt' .
\end{equation*}
In the version of the ct-fciqmc algorithm using exact rates the
corresponding probability would be
\begin{equation*}
R_{\bi\bj} e^{-R_{\bj}(t'-t)} dt' ,
\end{equation*}
with the exact total spawning rate $R_{\bj}$ appearing in the exponent
in place of the larger model rater $\tilde{R}_{\bj}$. The null-events
algorithm seems to be wrong.

But hold on. The null-events algorithm also includes failed or null
spawning events, which do not appear in the exact algorithm. To compare
the two algorithms properly, we must ask a slightly different question:
what is the probability that the first \emph{non-null} spawning event
attempted by the psip at $\bj$ creates a new psip on configuration $\bi$
in the time interval $[t',t'+dt']$? This successful spawning event may
have been preceded by any number of null events.

Consider, for example, the possibility that a null event might occur in
the interval $[t_1,t_1+dt_1]$, followed by the successful spawning of a
new psip on $\bi$ in the interval $[t',t'+dt']$. The probability of this
happening is
\begin{align*}
  \MoveEqLeft 
  ( \tilde{R}_{\bj} - R_{\bj} ) e^{-\tilde{R}_{\bj}(t_1 -
    t)} dt_1 \; \times \;
    R_{\bi\bj} e^{-\tilde{R}_{\bj}(t' - t_1)} dt' \\
    & = 
    ( R_{\bi\bj} \, dt' ) ( 
      \tilde{R}_{\bj} - R_{\bj} ) e^{-\tilde{R}_{\bj}(t'-t)} dt_1.
\end{align*}
Since the null event could occur at any time $t_1$ between $t$ and $t'$,
the total probability that a walker is spawned on $\bi$ after a single
null event is
\begin{align*}
\MoveEqLeft ( R_{\bi\bj} \, dt' ) 
( \tilde{R}_{\bj} - R_{\bj} ) e^{-\tilde{R}_{\bj}(t'-t)}
\int_{t}^{t'} dt_1 \\
& = ( R_{\bi\bj} \, dt' )
( \tilde{R}_{\bj} - R_{\bj} ) ( t'-t )
 e^{-\tilde{R}_{\bj}(t'-t)} .
\end{align*}

Now consider the more general case in which $n$ null events occur at
times $t_1 \leq t_2 \leq t_3 \ldots \leq t_n$, followed by the
successful spawning of a new walker on $\bi$ at time $t'$. The
probability of this happening is
\begin{align*}
\MoveEqLeft \int_{t}^{t'} dt_1 (\tilde{R}_{\bj} - R_{\bj})
e^{-\tilde{R}_{\bj}(t_1 - t)} \int_{t_1}^{t'} dt_2
(\tilde{R}_{\bj} - R_{\bj}) e^{-\tilde{R}_{\bj}(t_2 - t_1)} \\
& \ldots \int_{t_{n-1}}^{t'} dt_n
(\tilde{R}_{\bj} - R_{\bj}) e^{-\tilde{R}_{\bj}(t_n - t_{n-1})}
\times R_{\bi\bj} e^{-\tilde{R}_{\bj}(t'-t_n)} dt' \\
& = ( R_{\bi\bj} \, dt' ) ( \tilde{R}_{\bj} - R_{\bj}
)^{n} e^{-\tilde{R}_{\bj}(t'-t)} \int_{t \leq t_1 \leq t_2 \ldots \leq
  t_n \leq t'} dt_1 dt_2 \ldots dt_n.
\end{align*}
The $t_1 \leq t_2 \leq \ldots \leq t_n$ ordering of the time variables
in the $n$-dimensional integral is only one of $n!$ possible orderings,
each of which would yield the same result. Moreover, if the integrals
corresponding to all $n!$ possible orderings were added together, one
would obtain an integral over the $n$-dimensional hyper-cube $t \leq t_1
\leq t'$, $t \leq t_2 \leq t'$, $\ldots$, $t \leq t_n \leq t'$.  Since
the volume of the hyper-cube is $(t'-t)^{n}$, the time-ordered integral
must yield $(t'-t)^n/n!$. The probability that a new walker is
successfully spawned on $\bi$ in the time interval $[t',t'+dt']$ after
$n$ failed spawning events is thus
\begin{equation*}
(R_{\bi\bj}\,dt')  \frac{(\tilde{R}_{\bj} - R_{\bj})^n (t'-t)^n}{n!} 
e^{-\tilde{R}_{\bj}(t'-t)} .
\end{equation*}
The total probability that the first successful spawning event creates a
new psip on $\bi$ in the time interval $[t',t'+dt']$ (regardless of how
many failed spawning attempts precede the success) is obtained by
summing this result over all values of $n$ from zero to infinity. This
yields
\begin{align*}
\MoveEqLeft (R_{\bi\bj}\,dt') e^{-\tilde{R}_{\bj}(t'-t)} \sum_{n=0}^{\infty}
\frac{[(\tilde{R}_{\bj} - R_{\bj})(t'-t)]^n}{n!} \\
& = (R_{\bi\bj}\,dt') e^{-\tilde{R}_{\bj}(t'-t)}
e^{(\tilde{R}_{\bj}-R_{\bj})(t'-t)} \\
& = R_{\bi\bj} e^{-R_{\bj}(t'-t)} dt' .
\end{align*}
The probability that the first \emph{successful} spawning event creates
a new psip on configuration $\bi$ in the time interval $[t',t'+dt']$ is
therefore just $R_{\bi\bj} e^{-R_{\bj}(t'-t)} dt'$, exactly as in the
exact algorithm. As long as null events are ignored, the
model-rates/null-event algorithm yields exactly the right spawning
dynamics!

A much easier route to the same result can be found by going back to the
argument on page 2, which was used to derive the Poisson distribution of
spawning times for the \emph{exact} ct-fciqmc algorithm. Splitting the
evolution into time steps $\Delta t$, we asked: ``What is the
probability, $p_{\bi}$, that the first child spawned appears on
determinant $\bi$ during the time step starting at time $t' = t +
n\Delta t$?''  The answer was:
\begin{align*}
\begin{split}
p_\bi =& 
\underbrace{\prod_{\bk} (1-R_{\bk\bj}\Delta
  t)}_{\textrm{\parbox{4cm}{prob.\ \emph{no} children spawned in 1st
      time step}}} \times
\quad\underbrace{\prod_{\bk} (1-R_{\bk\bj}\Delta t)}_{\textrm{\parbox{4cm}{prob.\ \emph{no} children spawned in 2nd time step}}}
\cdots \\
& \times \underbrace{\prod_{\bk} (1-R_{\bk\bj}\Delta t)}_{\textrm{\parbox{4cm}{prob.\ \emph{no} children spawned in $n$-th time step}}} 
\quad \times \underbrace{R_{\bi\bj}\Delta t}_{\textrm{\parbox{4cm}{\rule{0mm}{5.5mm}prob.\ child spawned on determinant $\bi$ in $(n+1)$-th time step}}}
\end{split} \\
\hphantom{p}=& \prod_{\bk} (1-R_{\bk\bj}\Delta t)^n R_{\bi\bj}\Delta
t \\
\hphantom{p}\approx& \prod_{\bk}
\left(1-\frac{R_{\bk\bj}(t'-t)}{n}\right)^n 
R_{\bi\bj}\Delta t \\
\hphantom{p}\approx&  \exp\left(-\sum_\bk R_{\bk\bj}(t'-t)\right)
R_{\bi\bj}\Delta t. \\
\hphantom{p}=& \exp\left(- R_{\bj}(t'-t)\right)
R_{\bi\bj}\Delta t,
\end{align*}
where the last few lines require $\Delta t$ sufficiently small and $n$
sufficiently large. To deal with the null-events algorithm, we simply
imagine adding a new process, the null-spawning process, with rate
$\tilde{R} - R$. The clever part is that the addition of the
null-spawning process has \emph{absolutely no effect} on the argument
above: the expression for the probability that \emph{no} children are
spawned in a given time step is unaltered, as is the probability that a
child \emph{is} spawned on configuration $\bi$ in the final time
step. Seen from this viewpoint, it is pretty obvious that the
null-events algorithm can only produce the same spawning dynamics as the
exact ct-fciqmc algorithm.


\appendix
\section{Evaluation of $R_{\bi\bj}$}

An efficient continuous-time FCIQMC algorithm requires efficient enumeration of all the possible excitations from determinant $\bj$ and the associated $R_{\bi\bj}$ matrix elements.  Fortunately, this is relatively straight-forward in the Hubbard model (and also in, e.g., the uniform electron gas).

\subsection{Real space spin-orbitals}

If the determinant space is constructed from the real space (atomic) spin-orbitals, then only single excitations are connected and $R_{\bi\bj}$ has only three possible values:
\begin{equation}
R_{\bi\bj} = \begin{cases} 
    R_{\bj\bj} & \bi = \bj \\
    t          & \textrm{\parbox{9cm}{$|D_\bi\ket=a^\dagger_a a_i|D_\bj\ket$, where $i$ and $a$ are respectively occupied and unoccupied spin-orbitals on neighbouring sites}} \\
    0          & \textrm{otherwise.}
\end{cases}
\end{equation}
Due to the form of the Hubbard Hamiltonian, the maximum number of excitations possible from a determinant is $2n_\textrm{dim}N$, where $n_\textrm{dim}$ is the number of dimensions in the system and $N$ is the number of electrons.  The number of possible excitations is often reduced by orbital $a$ already being occupied.

\subsection{Bloch spin-orbitals}

In contrast, if the determinant is constructed from the Bloch spin-orbitals, then only double excitations are connected.  Again, $R_{\bi\bj}$ has three possible values:
\begin{equation}
R_{\bi\bj} = \begin{cases} 
    R_{\bj\bj} & \bi = \bj \\
    U          & \textrm{\parbox{9cm}{$|D_\bi\ket=a^\dagger_a a^\dagger_b a_i a_j|D_\bj\ket$, where $i,j$ and $a,b$ are respectively occupied and unoccupied spin-orbitals and conserve crystal momentum.}} \\
    0          & \textrm{otherwise.}
\end{cases}
\end{equation}
Crystal momentum is conserved if 
\begin{equation}
\bk_i + \bk_j = \bk_a + \bk_b,
\end{equation}
where here $\bk_i$ refers to the wavevector of the $i$-th spin-orbital.

Due to the form of the Hubbard Hamiltonian, only double excitations where $i$ and $j$ are of opposite spin are allowed and the selection of the fourth spin-orbital involved in a connected excitation is uniquely determined by the other three spin-orbitals.  Thus the maximum number of excitations from a determinant is $N_\uparrow N_\downarrow \min( M - N_\uparrow, M - N_\downarrow)$, where $N_\alpha$ is the number of electrons of spin $\alpha$ and $2M$ is the total number of spin-orbitals.

\end{document}
